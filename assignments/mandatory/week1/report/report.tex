\documentclass[a4paper, 10pt, final]{article}
\usepackage{bonde}

\def\mytitle{Signal and Image Processing 2010}
\def\mysubtitle{Handin of mandatory excercise 1}
\def\myauthor{Ulrik Bonde}
\def\mymail{\mailto{bonde@diku.dk}}
\def\mydate{\today}

\title{\mytitle}
\subtitle{\mysubtitle}

\author{\myauthor{} - \mymail}
\date{\mydate}

\hypersetup{
colorlinks,%
citecolor=black,%
filecolor=black,%
linkcolor=black,%
urlcolor=black,%
bookmarksopen=false,
pdftitle={\mytitle - \mysubtitle},
pdfauthor={\myauthor}
}

\begin{document}
\maketitle

\subsection*{Question 1.1}
We wish to prove that the Fourier transform of a real and even function
is real and even. We solve the integral for the Fourier transform in
equation (2.18) \citet[p. 44]{jahne-digital}.

\begin{equation*}
    \hat{g}(k) = \int_{-\infty}^{\infty}{g(x)\exp(-2\pi ikx)dx}
\end{equation*}

We split the integral in positive and negative parts, flip the
boundaries of the integral and perform the substitution $x = -x$. We can
do this substitution as we have an even function thus $g(x) = g(-x)$.

\begin{align*}
    \hat{g}(k) & = \int_{-\infty}^{0}{g(x)\exp(-2\pi ikx)dx} + \int_{0}^{\infty}{g(x)\exp(-2\pi ikx)dx}\\
               & = \int_{0}^{\infty}{g(x)\exp(-2\pi ikx)dx} - \int_{0}^{-\infty}{g(x)\exp(-2\pi ikx)dx}\\
               & = \int_{0}^{\infty}{g(x)\exp(-2\pi ikx)dx} + \int_{0}^{\infty}{g(-x)\exp(2\pi ikx)dx}
\end{align*}

Again we use that $g(x) = g(-x)$ and we can now integrate from
$0$ to $\infty$.

\begin{align*}
    \hspace{0.5em} &= \int_{0}^{\infty}{g(x)\exp(-2\pi ikx)dx} + \int_{0}^{\infty}{g(x)\exp(2\pi ikx)dx}\\
               & = \int_{0}^{\infty}{g(x)\left(\exp(-2\pi ikx) + \exp(2\pi ikx)\right)dx}
\end{align*}

By \emph{Euler's identities} we have that $\cos(\theta) =
\frac{e^{i\theta} + e^{-i\theta}}{2}$. We set $\theta = 2\pi kx$ and
multiply by $2$.

\begin{align*}
    \hspace{0.5em} & = 2 \int_{0}^{\infty}{g(x)\frac{\exp(-2\pi ikx) + \exp(2\pi ikx)}{2}dx}\\
               & = 2 \int_{0}^{\infty}{g(x)\cos(2\pi kx)dx}
\end{align*}

In the resulting $\hat{g}(k) = 2 \int_{0}^{\infty}{g(x)\cos(2\pi kx)dx}$
we have removed the imaginary part. The result is real and even because
$\cos(x)$ is an even function.

\subsection*{Question 1.2}
First we would like to find the Fourier transform of $\delta_{x - x_0} +
\delta_{x + x_0}$ (using the notation from \citeauthor{jahne-digital} for
$\delta$-functions). We set $g(x) = \delta_{x - x_0} + \delta_{x + x_0}$
and use equation (2.18) from \citeauthor{jahne-digital}.

\begin{align*}
    \hat{g}(k) & = \int_{-\infty}^{\infty}{g(x)\exp(-2\pi ikx)dx}\\
    & = \int_{-\infty}^{\infty}{(\delta_{x - x_0} + \delta_{x + x_0})e^{-2\pi ikx}dx}\\
    & = \int_{-\infty}^{\infty}{\delta_{x - x_0}e^{-2\pi ikx}dx +
    \int_{-\infty}^{\infty}{\delta_{-x - x_0}e^{-2\pi ikx}dx}}\\
\end{align*}
Note that we have used that $x + x_0 = -x - x_0$. In the second integral
we want to substitute $-x$ with $x$ to change signs in the exponential
function. To do this we have to multiply by $-1$ which changes the sign
of the integral. Also, the boundaries are reversed. We fix this by
changing the sign once again.
\begin{align*}
    & = \int_{-\infty}^{\infty}{\delta_{x - x_0}e^{-2\pi ikx}dx - \int_{\infty}^{-\infty}{\delta_{x - x_0}e^{2\pi ikx}dx}}\\
    & = \int_{-\infty}^{\infty}{\delta_{x - x_0}e^{-2\pi ikx}dx + \int_{-\infty}^{\infty}{\delta_{x - x_0}e^{2\pi ikx}dx}}
\end{align*}
When we integrate from $-\infty$ to $\infty$ there is only one point
where $x = x_0$ which means that the exponential functions in the
integral only contribute once. We then have that
\begin{align*}
    & = e^{-2\pi ikx} + e^{2\pi ikx}\\
    & = 2 \left(\frac{e^{-2\pi ikx} + e^{2\pi ikx}}{2}\right)\\
    & = 2 \cos(2\pi kx)
\end{align*}

By equation (2.19) from \citeauthor{jahne-digital} we have that
\begin{align*}
    g(x) & = \int_{-\infty}^{\infty}{\hat{g}(k)e^{2\pi ikx}dk}
\end{align*}
which will return the original function given the Fourier transform of
$g(x)$. Based on the above we guess that the Fourier transform of
$\cos(2\pi xu_0)$ is $\frac{1}{2}(\delta_{x - x_0} + \delta_{x + x_0})$.
\begin{align*}
    g(x) & = \int_{-\infty}^{\infty}{\frac{1}{2}(\delta_{x - x_0} +
    \delta_{x + x_0})e^{2\pi iu_0x}du_0}\\
    & = \frac{1}{2}\left( \int_{-\infty}^{\infty}{\delta_{x -
    x_0}e^{2\pi iu_0x}du_0} + \int_{-\infty}^{\infty}{\delta_{x +
    x_0}e^{2\pi iu_0x}du_0}\right)\\
    & = \cos(2\pi xu_0)
\end{align*}
We derive the result in a similar manner as previously except we
perform the substitution $u_0 = - u_0$. We use that the second
$\delta$-function can be written as $\delta_{x - (-x_0)}$ which then
again only contributes once.

\subsection*{Question 1.3}
The frequency of bass signals are low while treble signals have high
frequency.

\subsection*{Question 1.4}



%%%%%%%%%%%%%%%%%%%%%%%%%%%%%%%%%%%%%%%%%%%%%%%%%%%%%%%%%%%%%%%%%%%%
% Formal stuff

\bibliographystyle{abbrvnat}
\bibliography{bibliography}
%\addcontentsline{toc}{chapter}{Litteratur}

\end{document}

% vim: set tw=72 spell spelllang=en:
