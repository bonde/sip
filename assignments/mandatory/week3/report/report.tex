\documentclass[a4paper, 10pt, final]{article}
\usepackage{bonde}

\def\mytitle{Signal and Image Processing 2010}
\def\mysubtitle{Handin of mandatory excercise 3}
\def\myauthor{Ulrik Bonde}
\def\mymail{\mailto{bonde@diku.dk}}
\def\mydate{\today}

\title{\mytitle}
\subtitle{\mysubtitle}

\author{\myauthor{} - \mymail}
\date{\mydate}

\hypersetup{
colorlinks,%
citecolor=black,%
filecolor=black,%
linkcolor=black,%
urlcolor=black,%
bookmarksopen=false,
pdftitle={\mytitle{} - \mysubtitle},
pdfauthor={\myauthor}
}

\begin{document}
\maketitle

\subsection*{Question 3.1}

\subsection*{Question 3.2}
We let $M = 2^m$ and $N = 2^n$ and let $I$ be a $M \times M$ image and
$f$ a seperable $N \times N$ filter. In the following we assume that
$M \geq N$.

The naive convolution $(f \star I)(x)$, disregarding that $f$ is separable,
will yield $M^2N^2$ multiplications, as we for every pixel in $I$ will
have to multiply that with every pixel in the filter. This is infeasible
for large values of $N$ (remember that $M \geq N$).

Using that $f$ is separable we can split the filter and convolute the
rows and columns of $I$ separably. The image has $M$ rows. For each row
in $I$ we have $M$ pixels. The filter have been reduced to a 1D filter
of length $N$. Then for each pixel in a row we perform $N$
multiplications. The same is true for the columns in $I$, i.e.  $M$
columns with $M$ pixels which must be multiplied $N$ times. The total
amount of calculations needed for convolution in the spatial domain
$C_S$ is derived below.
\begin{align}
    C_S & = M(MN) + M(MN)\\
    & = 2M^2N
\end{align}

Now we want to find the number of multiplications needed in the
frequency domain. To do this we must Fourier transform both $I$ and
$f$, but we should also make sure that the size of $f$ is equal to the
size of $I$ (to do pointwise multiplication in the frequency domain). We
pad $f$ with zeroes to match the size of $I$.

The Fourier transform is separable and again we can transform a single
row (or column) with $L\log(L)$ multiplications. The image have $M$ rows
and each row require $M\log(M)$ multiplications. The same is true for
the columns. The filter require the same amount. We then get a total of
$2(2(M^2\log(M)))$ multiplications just for the Fourier transform of $I$
and $f$. We assume that the inverse Fourier transform use the same
amount of calculations, we also need $2(M^2\log(M))$ for transforming
$I$ back to the spatial domain. The actual convolution in the frequency
domain is just pointwise multiplication of $\mathcal{F}(I)$ and
$\mathcal{F}(f)$. This is $M^2$ multiplications. The total amount of
multiplications in the frequency domain is then:
\begin{align}
    C_{\mathcal{F}} & = 2(2(M^2\log(M))) + M^2 + 2(M^2\log(M))\\
    & = 6M^2\log(M) + M^2
\end{align}
It should be noted that the multiplications in the frequency domain are
complex multiplications.

When should we use convolution in the frequency domain rather than in
the spatial domain? This depends on the size of the filter. We solve the
inequality $C_{\mathcal{F}} < C_S$.
\begin{align}
    C_{\mathcal{F}} & < C_S\\
    6M^2\log(M) + M^2 & < 2M^2N\\
    3\log(M) + \frac{1}{2} & < N
\end{align}
When the filter size $N$ is greater than $3\log(M) + 1/2$ then fewer
multiplication will have to be done if we transform to the Fourier
domain and do the convolution there.

If $\mathcal{F}(f)$ is already known, we can cut away the
multiplications for the transform of $f$, thus only requiring
$4M^2\log(M) + M^2$ multiplications. For convolution in the frequency
domain to be feasible we then only require that $N > 2\log(M) + 1/2$.


In table \ref{table_sizes} we see how this affects a real image with
$M = 512$.
\begin{table}[!h]
    \centering
    \begin{tabular}{|c|r|c|c|}
        \hline
        $N$ & $C_S$ & $C_{\mathcal{F}}$ & $C_{\mathcal{F}}~'$\\\hline
        $4$ & $2.097.152$ & $14.417.920$ & $9.699.328$\\
        $8$ & $4.194.304$ & $14.417.920$ & $9.699.328$\\
        $16$ & $8.388.608$ & $14.417.920$ & $9.699.328$\\
        $32$ & $16.777.216$ & $14.417.920$ & $9.699.328$\\
        \hline
    \end{tabular}
    \caption{Image with $M = 512$. When the filter size reaches $32 \times
    32$ it is no longer feasible to do the convolution in the spatial
    domain.}
    \label{table_sizes}
\end{table}

\subsection*{Question 3.3}


%%%%%%%%%%%%%%%%%%%%%%%%%%%%%%%%%%%%%%%%%%%%%%%%%%%%%%%%%%%%%%%%%%%%
% Formal stuff

\bibliographystyle{abbrvnat}
\bibliography{bibliography}
%\addcontentsline{toc}{chapter}{Litteratur}

\end{document}

% vim: set tw=72 spell spelllang=en:
