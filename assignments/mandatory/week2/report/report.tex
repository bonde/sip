\documentclass[a4paper, 10pt, final]{article}
\usepackage{bonde}

\def\mytitle{Signal and Image Processing 2010}
\def\mysubtitle{Handin of mandatory excercise 2}
\def\myauthor{Ulrik Bonde}
\def\mymail{\mailto{bonde@diku.dk}}
\def\mydate{\today}

\title{\mytitle}
\subtitle{\mysubtitle}

\author{\myauthor{} - \mymail}
\date{\mydate}

\hypersetup{
colorlinks,%
citecolor=black,%
filecolor=black,%
linkcolor=black,%
urlcolor=black,%
bookmarksopen=false,
pdftitle={\mytitle{} - \mysubtitle},
pdfauthor={\myauthor}
}

\begin{document}
\maketitle

\subsection*{Question 2.1}

\subsection*{Question 2.2}
We wish to find the Fourier transform $\hat{G}_{\sigma}(k)$ of the
gaussian distribution $G_{\sigma}(x)$.
\begin{align}
    \hat{G}_{\sigma}(k) & = \int_{-\infty}^{\infty}{\frac{1}{\sqrt{2\pi \sigma^{2}}}e^{-\frac{x^2}{2\sigma^2}}e^{-2\pi ikx}dx}\\
    & = \frac{1}{\sqrt{2\pi \sigma^{2}}}\int_{-\infty}^{\infty}{e^{-(\frac{1}{2\sigma^2}x^2 + 2\pi ikx)}dx}\\
    & = \frac{1}{\sqrt{2\pi \sigma^{2}}}\sqrt{\pi/\frac{1}{2\sigma^{2}}}e^{(2\pi ik)^2}/\frac{4}{2\sigma^2}\\
    & = e^{-2(\pi k\sigma)^2}
\end{align}
A lot things is going on. First, from (3) to (4), we have used that
\begin{equation}
    \int_{-\infty}^{\infty}e^{-(ax^2+bx+c)} = \sqrt{\pi/a}e^{b^2 - 4ac/4a}
\end{equation}
where $a = \frac{1}{2\sigma^2}$, $b = 2\pi ik$ and $c = 0$. From
(4) to (5) we rewrite some fractions. In (4) we have that
\begin{equation}
    \sqrt{\pi/\frac{1}{2\sigma^{2}}} =
    \sqrt{\pi\frac{2\sigma^2}{1}} = \sqrt{2\pi\sigma^2}
\end{equation}
which then eliminates the first term. We also have that
\begin{equation}
    (2\pi ik)^2 / \frac{4}{2\sigma^2} = \frac{-8\pi^{2}k^{2}\sigma^2}{4} = -2\pi^2 k^2\sigma^2
\end{equation}
where we notice the change in sign because $i^2 = -1$.

Just to recap, the Fourier transform of $G_{\sigma}(x)$ is
\begin{align}
    \hat{G}_{\sigma}(k) & = e^{-2(\pi k\sigma)^2}
\end{align}

We define 
\begin{align}
    h_1(x) & = G_{\sigma}(x)\cos(2\pi u_0x)\\
    h_2(x) & = G_{\sigma}(x)\sin(2\pi u_0x)\\
    h(x) & = h_1(x) + ih_2(x)
\end{align}
and we want to find the Fourier transform $\hat{h}(k)$ of $h(x)$.

Now the know $\hat{G}_{\sigma}(k)$ as well as the Fourier transforms of
both $\cos(2\pi u_0 x)$ and $\sin(2\pi u_0 x)$. The convolution theorem
says that convolution of two functions in the Fourier domain means
multiplication in the space domain. Therefore, instead of multiplying
the functions and then finding the Fourier transform, we just convolve
their Fourier transforms.
\begin{align}
    \hat{h}(x) & = \hat{G}_{\sigma}(k) \star \hat{h_1}(k) + i\hat{G}_{\sigma}(k) \star \hat{h_2}(k)\\
     & = \hat{G}_{\sigma}(k) \star (\hat{h_1}(k) + i\hat{h_2}(k))\\
     & = \hat{G}_{\sigma}(k) \star \left(\frac{\delta_{x - x_0} + \delta_{x + x_0}}{2} + i\frac{i\delta_{x - x_0} - \delta_{x + x_0}}{2}\right)\\
     & = \hat{G}_{\sigma}(k) \star \left(\frac{\delta_{x - x_0} + \delta_{x + x_0}}{2} - \frac{\delta_{x - x_0} - \delta_{x + x_0}}{2}\right)\\
     & = \hat{G}_{\sigma}(k) \star \delta_{x + x_0}\\
     & = \int_{-\infty}^{\infty}{e^{-2(\pi\sigma u_0)^2}\delta_{x + u_0}du_0}\\
     & = e^{-2(\pi\sigma u_0)^2}
\end{align}

\subsection*{Question 2.3}
Given two continuous signals $f(x)$ and $g(x)$ we wish to find the
convolution $f \star g$. The two signals are shown in fig.
\ref{signals}.

\begin{figure}[!h]
    \centering
    \subfloat[$f(x)$]{\mbox{
    \begin{picture}(150,70)
        \put(60, 47){$A$}
        \put(105, 0){$1$}
        %\put(0, 45){\circle*{3}}

        {
        \color{red}
        \linethickness{1.6pt}
        \put(70, 50){\line(1, 0){35}}
        \put(0, 15){\line(1, 0){70}}
        \put(105, 15){\line(1, 0){31}}
        \put(70, 15){\line(0, 1){35}}
        \put(105, 15){\line(0, 1){35}}
        }

        \put(0, 15){\vector(1, 0){140}}
        \put(70, 15){\vector(0, 1){50}}

    \end{picture}}
    }\hspace{1em}
    \subfloat[$g(x)$]{\mbox{
    \begin{picture}(150,70)
        \put(25, 47){$B$}
        \put(105, 0){$1$}
        \put(30, 0){$-1$}
        %\put(0, 45){\circle*{3}}

        {
        \color{blue}
        \linethickness{1.6pt}
        \put(35, 50){\line(1, 0){70}}
        \put(0, 15){\line(1, 0){35}}
        \put(105, 15){\line(1, 0){31}}
        \put(35, 15){\line(0, 1){35}}
        \put(105, 15){\line(0, 1){35}}
        }

        \put(0, 15){\vector(1, 0){140}}
        \put(70, 15){\vector(0, 1){50}}

    \end{picture}}
    }
    \caption{Two signals}
    \label{signals}
\end{figure}

Convolution of two continuous signals is defined as
\begin{equation}
    (g \star f)(x) = \int_{-\infty}^{\infty}{f(\alpha)g(x - \alpha)d\alpha}
\end{equation}
where we have used that convolution is commutative. The signals are only
defined on certain intervals, thus we need to find these integration
limits to perform convolution. We also note that $A$ and $B$ only are
constants, thus we can replace them by $1$ and multiply later.

\begin{figure}[!h]
    \centering
    \subfloat[$(g \star f)(x)$]{\mbox{
    \begin{picture}(150,70)
        \put(-5, 47){$1$}
        \put(40, 65){$\alpha$}
        \put(105, 0){$1$}
        \put(30, 0){$-1$}
        %\put(0, 45){\circle*{3}}

        {
        \color{red}
        \linethickness{1.6pt}
        \put(75, 50){\line(1, 0){35}}
        \put(5, 15){\line(1, 0){70}}
        \put(110, 15){\line(1, 0){31}}
        %\put(70, 15){\line(0, 1){35}}
        \put(110, 15){\line(0, 1){35}}
        }

        {
        \color{blue}
        \put(35, 10){\line(0, 1){45}}
        \linethickness{1.6pt}
        \put(0, 50){\line(1, 0){70}}
        \put(-10, 15){\line(1, 0){10}}
        \put(75, 15){\line(1, 0){31}}
        \put(0, 15){\line(0, 1){35}}
        \put(70, 15){\line(0, 1){35}}
        }

        \put(0, 15){\vector(1, 0){140}}
        \put(70, 15){\vector(0, 1){50}}

    \end{picture}}
    }\hspace{1em}
    \subfloat[$(g \star f)(x)$]{\mbox{
    \begin{picture}(150,70)
        \put(60, 47){$1$}
        \put(145, 65){$\alpha$}
        \put(105, 0){$1$}
        \put(145, 0){$2$}
        \put(30, 0){$-1$}
        %\put(0, 45){\circle*{3}}

        {
        \color{red}
        \linethickness{1.6pt}
        \put(75, 50){\line(1, 0){35}}
        \put(5, 15){\line(1, 0){70}}
        \put(110, 15){\line(1, 0){31}}
        \put(75, 15){\line(0, 1){35}}
        \put(110, 15){\line(0, 1){35}}
        }

        {
        \color{blue}
        \put(140, 10){\line(0, 1){45}}
        \linethickness{1.6pt}
        \put(105, 50){\line(1, 0){70}}
        \put(-100, 15){\line(1, 0){10}}
        \put(175, 15){\line(1, 0){31}}
        \put(105, 15){\line(0, 1){35}}
        \put(175, 15){\line(0, 1){35}}
        }

        \put(0, 15){\vector(1, 0){140}}
        \put(70, 15){\vector(0, 1){50}}

    \end{picture}}
    }
    \caption{Convolution of two signals}
    \label{signals_outer}
\end{figure}
We mirror $g(x)$ and ``slide'' it over $f(x)$ and from fig.
\ref{signals_outer} we see that the outer limits of the integral are
$\alpha < -1$ and $\alpha \geq 2$, as the signals do not interfere with
each other. Thus we have
\begin{equation}
    \int_{-\infty}^{-1}{f(\alpha)g(x - \alpha)d\alpha} =
    \int_{2}^{\infty}{f(\alpha)g(x - \alpha)d\alpha} = 0
\end{equation}
We see that if we move $g(x)$ further than $-1$ the signals overlap. We
have a new limit $-1 \leq \alpha < 0$ and we use the convolution
integral:
\begin{equation}
    AB\int_{x - 1}^{0}d\alpha = AB[\alpha]^{0}_{x - 1} = AB(-(x - 1)) =
    AB(-x + 1)
\end{equation}
Next, from 0 to 1, we can move $g(x)$ all the way across $f(x)$. We have
\begin{equation}
    AB\int_{0}^{1}d\alpha = AB[\alpha]^{1}_{0} = AB
\end{equation}
When $\alpha$ passes 1 the area of the intersection get linearly
smaller until 2, where the signal is zero for both. We integrate from
$x$ to 2:
\begin{equation}
    AB\int_{x}^{2}d\alpha = AB[\alpha]^{2}_{x} = 2AB-x
\end{equation}
The final convolution of $f(x)$ and $g(x)$ is then defined as
\begin{equation}
    (g \star f)(x) =
    \left\{ \begin{array}{l l r}
        AB(-x + 1) & \mbox{if} & -1 \leq x < 0\\
        AB & \mbox{if} & 0 \leq x < 1\\
        2AB - x & \mbox{if} & 1 \leq x < 2\\
        0 & \mbox{otherwise} & \\
    \end{array}
    \right.
\end{equation}
and the signal is shown in fig. \ref{signals_final}.
\begin{figure}[!h]
    \centering
    \mbox{
    \begin{picture}(150,70)
        \put(45, 47){$AB$}
        \put(135, 0){$2$}
        \put(30, 0){$-1$}

        {
        \color{green}
        \linethickness{1.6pt}
        \put(70, 50){\line(1, 0){35}}
        \put(35, 15){\line(1, 1){35}}
        \put(105, 50){\line(1, -1){35}}
        \put(5, 15){\line(1, 0){30}}
        \put(140, 15){\line(1, 0){31}}
        }

        \put(0, 15){\vector(1, 0){175}}
        \put(70, 15){\vector(0, 1){50}}

    \end{picture}}
    \caption{Signal of $(f \star g)(x)$}
    \label{signals_final}
\end{figure}

\subsection*{Question 2.4}

%%%%%%%%%%%%%%%%%%%%%%%%%%%%%%%%%%%%%%%%%%%%%%%%%%%%%%%%%%%%%%%%%%%%
% Formal stuff

%\bibliographystyle{abbrvnat}
%\bibliography{bibliography}
%\addcontentsline{toc}{chapter}{Litteratur}

\end{document}

% vim: set tw=72 spell spelllang=en:
